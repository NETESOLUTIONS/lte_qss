\documentclass[notitlepage]{report}
\usepackage[left=1in, right=1in, top=1in, bottom=1in]{geometry}
\usepackage{titling}
\usepackage{lipsum}
\pretitle{\begin{center}\Huge\bfseries}
\posttitle{\par\end{center}\vskip 0.5em}
\preauthor{\begin{center}\Large\ttfamily}
\postauthor{\end{center}}
%\predate{\par\large\centering}
%\postdate{\par}
\title{Higher-order Combinations of Existing Ideas: Scalable computing helps}
\author{Us and others}
%\date{\today}
\date{}
\begin{document}
\maketitle
\thispagestyle{empty}
Sir:\\

Marshakova-Shaikevich and Small independently described co-citation in 1973 and remarked on it being a forward looking measurement. Co-citation has been extensively used by the field since then and has also been used in large-scale studies (Uzzi, Boyack, Bradley). In 1973, Small reported 4 pairs of articles with a co-citation frequency of 49 and greater from the field of physics. Since then, improved bibliographies and modern computing tools have rendered co-citation calculations relatively facile. The volume of scientific literature has also grown since 1973; citations and co-citations have accrued in greater numbers. Devarakonda reported that X of 33.6 million co-cited pairs derived from articles published in Scopus in the years 1985--1995 had co-citation frequencies of over 1,000. More recently, Zhao et al. (under review) computed co-citation frequencies for 940 million article pairs from the same period and identified roughly 1200 pairs with delayed kinetics analogous to Sleeping Beauty publications. 

Document coupling of a higher order, triads, tetrads, and greater is a natural extension of  co-citations, and measuring these are now within the reach of the scientometrist. The first  higher-order construct beyond co-citation is a tri-cite or triad, which can be thought of as either the coming together of three independent ideas or the assembly of three overlapping co-citations. 

With affordable cloud computing and improved bibliographies, measuring these higher order representations of new ideas is now within the reach of the scientometrist. The magnitude of scale-up from co-citations to triads and tetrads alone can be appreciated in the following example. An article with 25 or 50 cited references represents 300 or 1,225 co-citations respectively. The same article also consists of  2,300 or 19,600 triads or 12,650 and  230,300 tetrads. For a modestly sized collection of 2 million articles with an average of 30 references each, $8.12 \times10^9$ triads would need to be generated and deduplicated after which once could compute their frequency of tri-citation. In the case of tetrads, this number increases to $5.48 \times 10^{10}$. 
Using a graph database, the Scopus bibliography, and cloud computing we have conducted an initial investigation of triads.
As a first step, we have examined triads- sets of three articles that are tri-cited. We  have selected triads in two different ways. (i) Restricting all triads generated from X milllion articles (Zhao) to those where all three members of the triad are in the top 1\% of cited articles in Scopus, and (ii) all triads where two member of the triad are a co-cited pair previously generated by Zhao et al. Unsurprisingly, there is overlap in the two sets. 

What does this all mean? 

\end{document}