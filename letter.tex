\documentclass[notitlepage]{report}
\usepackage[left=1in, right=1in, top=1in, bottom=1in]{geometry}
\usepackage{titling}
\usepackage{lipsum}
\pretitle{\begin{center}\Huge\bfseries}
\posttitle{\par\end{center}\vskip 0.5em}
\preauthor{\begin{center}\Large\ttfamily}
\postauthor{\end{center}}
%\predate{\par\large\centering}
%\postdate{\par}
\title{Higher-order Combinations of Existing Ideas}
\author{Us and others}
%\date{\today}
\date{}
\begin{document}
\maketitle
\thispagestyle{empty}
Sir:\\

Co-citation, when two articles are cited by a third, represents the combination of two existing ideas into a third. The frequency of co-citation, which accumulates over time, represents the extent to which this idea is considered by the research community. Marshakova-Shaikevich and Small independently described co-citation in 1973, and emphasized its forward looking characteristics compared to bibliographic coupling.  Co-citation measurements have since been extensively used in scientometrics for a variety of purposes.

From a sampling of the physics literature, Small reported 4 pairs of articles with a co-citation frequency of 49 and greater. The volume of scientific literature has grown considerably since; more co-cited pairs have been discovered; frequencies have increased; and the scale of studies has become larger. Improved bibliographies and modern computing tools have also rendered co-citation calculations relatively facile. 

Uzzi et al. reported a study on novelty in which they examined co-cited references from roughly 18 million articles. Devarakonda et al. reported that 1,309 of 33.6 million co-cited pairs derived from articles published in Scopus in the years 1985--1995 had co-citation frequencies of over 1,000. More recently, Zhao et al. have computed co-citation frequencies for 940 million co-cited article pairs from the same period and identified roughly 1,200 pairs with delayed kinetics analogous to Sleeping Beauty publications. 

A natural extension of co-citation theory is document coupling of a higher order, for example, triads, tetrads, and greater. Such calculations are relatively expensive, however. An article with either 25 or 50 cited references represents 300 or 1,225 $n\choose2$ co-citations respectively. The same article also consists of  2,300 or 19,600 tri-citations and  12,650 or  230,300 tetra-citations. This implies, for a modestly sized collection of 2 million articles with an average of 30 references each, computing $8.12 \times10^9$ tri-citations or $5.48 \times 10^{10}$ tetra-citations. Today, scalable and affordable computing places measuring these higher order assemblies within reach of the scientometrist, and opens a new frontier for discovery.

In an initial exploration of higher-order combinations, we have computed frequencies of triads (sets of three articles that are `tri-cited' by other articles) of cited references from a dataset of articles in Scopus published between in the years 1985--1995. Triads can be viewed as either the coming together of three previously independent ideas or the result of three overlapping co-citations in the triad. Thus, we also computed direct citations and bibliographic coupling for members of each of these triads using a graph database, the Scopus bibliography, and scalable cloud computing resources. \emph{This is where we insert Wenxi's data as a multi-panel figure...and get Henry's initial thoughts.}

\end{document}