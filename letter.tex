\documentclass[notitlepage]{report}
\usepackage[left=1in, right=1in, top=1in, bottom=1in]{geometry}
\usepackage{titling}
\usepackage{lipsum}
\pretitle{\begin{center}\Huge\bfseries}
\posttitle{\par\end{center}\vskip 0.5em}
\preauthor{\begin{center}\Large\ttfamily}
\postauthor{\end{center}}
%\predate{\par\large\centering}
%\postdate{\par}
\title{Data and Computing Enable Studying Higher-order Combinations of Existing Ideas}
\author{Us and others}
%\date{\today}
\date{}
\begin{document}
\maketitle
\thispagestyle{empty}
Sir:\\

Marshakova-Shaikevich and Small independently described co-citation in 1973 and remarked on it being a forward looking measurement. Co-citation has been extensively used by the field since then and has also been used in large-scale studies (Uzzi, Boyack, Bradley). The volume of scientific literature has grown since 1973; citations and co-citations have also accrued. In 1973, Small reported 4 pairs of articles with a co-citation frequency of 49 and greater from the field of physics. Since then, improved bibliographies and modern computing tools have rendered co-citation calculations relatively facile. Devarakonda reported that X of 33.6 million co-cited pairs derived from articles published in Scopus in the years 1985--1995 had co-citation frequencies of over 1,000. More recently, Zhao et al. (under review) computed co-citation frequencies for 940 million article pairs from the same period from which they identified roughly 1200 pairs with delayed kinetics that were analogous to Sleeping Beauty publications. 

Document coupling of a higher order, triads, tetrads, and larger are a logical extension of our experience with co-citation and now within the reach of the scientometrist. While studying them is still computationally expensive, they are beginning to come within the reach of the scientometrist. The scale-up from co-citations to triads can be appreciated by the following example. An article with 25 or 50 cited references provides 300 or 1,225 co-cited pairs respectively for analysis. The same article would result in 2,300 or 19,600 triads or 12,650 and  230,300 tetrads. For a modestly sized collection of 2 million articles with an average of 30 references each, $8.12 .10^9$ triads would need to be generated and deduplicated after which once could compute their frequency of tri-citation. In the case of tetrads, this number increases to $5.48.10^{10}$. 

As a first step, we have examined triads- sets of three articles that are tri-cited. Interestingly,  a triad can also be thought of as the union of three  co-cited pairs. We have used a graph database and the Scopus bibliography in an initial examination of triads. We  have selected triads in two different ways. (i) Restricting all triads generated from X milllion articles (Zhao) to those where all three members of the triad are in the top 1\% of cited articles in Scopus, and (ii) all triads where two member of the triad are a co-cited pair previously generated by Zhao et al. Unsurprisingly, there is overlap in the two sets. 

What does this all mean? 

\end{document}