\documentclass[notitlepage]{report}
\usepackage[left=1in, right=1in, top=1in, bottom=1in]{geometry}
\usepackage{titling}
\usepackage{lipsum}
\pretitle{\begin{center}\Huge\bfseries}
\posttitle{\par\end{center}\vskip 0.5em}
\preauthor{\begin{center}\Large\ttfamily}
\postauthor{\end{center}}
%\predate{\par\large\centering}
%\postdate{\par}
\title{Higher-order Combinations of Existing Ideas: Scalable computing helps}
\author{Us and others}
%\date{\today}
\date{}
\begin{document}
\maketitle
\thispagestyle{empty}
Sir:\\

Marshakova-Shaikevich and Small independently described co-citation in 1973 and emphasized its forward looking characteristics compared to bibliographic coupling. In 1973, Small reported 4 pairs of articles with a co-citation frequency of 49 and greater from his sampling of the field of physics. Since then, improved bibliographies and modern computing tools have rendered co-citation calculations relatively facile. Co-citation measurements have since been extensively used in scientometrics. The volume of scientific literature grown considerably since 1973; citations and co-citations have also accrued in greater numbers as have the scale of studies. Uzzi et al. reported a study on novelty in which they examined co-cited references from roughly 18 million articles. Devarakonda et al. reported that 1,309 of 33.6 million co-cited pairs derived from articles published in Scopus in the years 1985--1995 had co-citation frequencies of over 1,000. More recently, Zhao et al. computed co-citation frequencies for 940 million co-cited article pairs from the same period and identified roughly 1,200 pairs with delayed kinetics analogous to Sleeping Beauty publications. Thus, co-citation has been explored at scale.

A natural extension of co-citation theory is document coupling of a higher order, for example, triads, tetrads, and greater. Such calculations are computationally expensive, however, as the following example indicates. An article with either 25 or 50 cited references represents 300 or 1,225 $n\choose2$ co-citations respectively. The same article also consists of  2,300 or 19,600 triads and  12,650 or  230,300 tetrads. For a modestly sized collection of 2 million articles with an average of 30 references each, $8.12 \times10^9$ triads would need to be generated and deduplicated after which once could compute their frequency of tri-citation across a large research-grade bibliography. In the case of tetrads, this number increases to $5.48 \times 10^{10}$.  With affordable cloud computing and improved bibliographies, however, measuring these higher order representations of new ideas is now within the reach of the scientometrist, and presents a new frontier with opportunities for discovery.

In an initial exploration of higher-order combinations, we have computed frequencies of triads (sets of three articles that are `tri-cited' by other articles) of cited references from a dataset of articles in Scopus published between in the years 1985--1995. Triads can be viewed as either the coming together of three previously independent ideas or the product of interactions among members of the triad. Thus, we also computed direct citations and bibliographic coupling for members of each of these triads using a graph database, the Scopus bibliography, and scalable cloud computing resources. This is where we insert Wenxi's data as a multi-panel figure...

\end{document}